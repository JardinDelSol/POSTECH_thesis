%%
%% TeX:UTF-8
%%
%% POSTECH 학위논문양식 LaTeX용 (ver 0.4) 예시
%%
%% @version 0.6
%% @author  박준신 Junshin Park (mailto:lonelywing@postech.ac.kr)
%% @date    2021. 3. 24.
%%
%% @requirement
%% teTeX, fpTeX, teTeX 등의 LaTeX2e 배포판
%% + 은광희 님의 HLaTeX 0.991 이상 버젼 또는 홍석호 님의 HPACK 1.0
%% : 설치에 대한 자세한 정보는 http://www.ktug.or.kr을 참조바랍니다.
%% 
%% @Tested in
%% Windows 7, Texlive 2014.
%% Ubuntu 14.04 LTS, Texlive 2016.
%% Compatibility with miktex has not been tested yet.
%%
%% @note for international students
%% Present format is intended for both Korean and international students.
%% For those control statements or arguments that need guidance, 
%% comments are added in both languages for convenience. You may however delete the Korean comments 
%% if your specific environment does not support Korean.
%% For more information or help, please contact the author of which email address is written above.
%%
%%
%% @acknowledgement
%% 본 latex template은 kaist 채승병님의 template을 바탕으로 만들어졌음을 밝힙니다.
%% Alpha test 및 본 thesis 예시 내용은 컴퓨터공학과 졸업생이신 오수영 학우의 도움으로 작성되었습니다.
%% 또한, Sae Hee Ryu의 지원이 한 스푼 들어갔습니다.
%% -------------------------------------------------------------------
%% @information
%% 이 예제 파일은 hangul-ucs를 사용합니다. UTF-8 입력 인코딩으로
%% 작성되었습니다. hlatex의 hfont는 이용하지 않습니다. --2006/02/11



% @class postech-ucs.cls
% @options [default: doctor, korean, final]
% - doctor: 박사과정 | master : 석사과정
% - korean: 한글논문 | english: 영문논문
% - final : 최종판   | draft  : 시험판
% - pdfdoc : 선택하지 않으면 북마크와 colorlink를 만들지 않습니다.(Generate bookmark and colorlink if enabled)
\documentclass[doctor,english,final]{postech-ucs}
%\documentclass[master,english,final]{postech-ucs}
% If you want make pdf document (include bookmark, colorlink)
%\documentclass[doctor,english,final,pdfdoc]{postech-ucs}

% postech-ucs.cls 에서는 기본으로 dhucs, ifpdf, graphicx, afterpage, refcount 패키지가 로드됩니다.
% (postech-ucs.cls by default loads dhucs, ifpdf,refcount and graphicx packages. Please load any additional packages if needed)
% 추가로 필요한 패키지가 있다면 주석을 풀고 적어넣으십시오,
%\usepackage{...}
\usepackage{amsmath}
\usepackage{enumitem}
\usepackage{algorithm}
\usepackage{algpseudocode}
\usepackage{comment}
\usepackage{physics}
\usepackage{anyfontsize}
% \usepackage{bbm}
% \renewcommand{\thealgorithm}{}
\newcommand{\argmax}{\operatornamewithlimits{arg\,max}}

% @command title 논문 제목(title of thesis)
% @options [default: (none)]
% - korean: 한글제목(korean title) | english: 영문제목(english title)
\title[korean] {각분해광전자분광법을 이용한 표면 도핑된 흑린의 전자구조에 관한 연구}
\title[english]{Band structure of doped black phosphorus studied by angle-resolved photoemission spectroscopy}

% @note 표지에 출력되는 제목을 강제로 줄바꿈하려면 \linebreak 을 삽입.
%       \\ 나 \newline 등을 사용하면 안됩니다. (아래는 예시)
%
% If you want to begin a new line in cover, use \linebreak .
% See examples above.
%


% @command author 저자 이름
% @param   family_name, given_name 성, 이름을 구분해서 입력
% @options [default: (none)]
% - korean: 한글이름 | chinese: 한문이름 | english: 영문이름
% ex) \author[english]{family name in english}{given name in english}
%
% If you are a foreigner (this means you have no korean name),
% You must fill the korean name as blank, instead of deleting it or commenting it out.
% \author[korean] {}{}
% \author[english]{Donald}{Trump}
%
%
\author[korean] {류}{세 희}
\author[english]{Ryu}{Sae Hee}

% @command advisor 지도교수 이름 (복수가능)
% @usage   \advisor[options]{...한글이름...}{...영문이름...}{signed|nosign}
% @options [default: major]
% - major: 주 지도교수  | coopr: 공동 지도교수
\advisor[major]{박 재 훈}{Jae-Hoon Park}{nosign}
\advisor[coopr]{김 근 수}{Keun Su Kim}{nosign}
%
% 지도교수 한글이름은 입력하지 않아도 됩니다.
% You may not input advisor's korean name
% like this \advisor[major]{}{Chang, Kee Joo}{signed}
%

% @command department {학과이름}{학위종류} - 아래 표에 따라 코드를 입력
% @command department {department code}{degree field}
%
% department code table
%
% MA	// 수학 	Department of Mathematics
% PH	// 물리학 Department of Physics 
% CH 	// 화학 	Department of Chemistry
% LS  	// 생명과학 Department of Life Sciences
% MS 	// 신소재공학 Department of Materials Science and Engineering
% ME 	// 기계공학 Department of Mechanical Engineering
% EE 	// 전자전기공학 Department of Electrical Engineering
% IME 	// 산업경영공학 Department of Industrial & Management Engineering
% CSE 	// 컴퓨터공학 Department of Computer Science and Engineering
% CE 	// 화학공학	Department of Chemical Engineering
% CITE 	// 창의IT	Department of Creative IT Excellence Engineering
% AMS	// 첨단재료과학 Division of Advanced Material Science
% IBB 	// 시스템생명공학 Division of Integrative Bioscience and Biotechnology 
% ITCE 	// 정보전자융합 Division of IT Convergence Engineering
% ANE 	// 첨단원자력공학 Division of Advanced Nuclear Engineering
% EV 	// 환경공학	School of Environmental Science and Engineering
% IBT	// 융합생명공학과 School of Interdisciplinary Bioscience and Bioengineering
% TIM 	// 기술경영공학	Graduate Program for Technology & Innovation Management
% WE 	// 풍력특성화 School of Wind Energy
% GEM 	// 엔지니어링특성화 Graduate School of Engineering Mastership
% GIFT 	// 철강대학원 Graduate Institute of Ferrous Technology
% OSTI 	// 해양대학원 POSTECH Ocean Science & Technology Institute

%
% science: 이학 | engineering: 공학 | business : 경영학
% 박사논문의 경우는 학위종류를 입력하지 않아도 됩니다.
% If you write Ph.D. dissertation, you cannot input degree field.

\department{PH}{}

% @command studentid 학번(ID)
\studentid{20131187}

% @command referee 심사위원 (석사과정 3인, 박사과정 5인)
\referee[1]{Jae-Hoon Park}
\referee[2]{Myung Ho Kang}
\referee[3]{Jeehoon Kim}
\referee[4]{Tae-Hwan Kim}
\referee[5]{Keun Su Kim}
% Of course english name is available

% @command approvaldate 지도교수논문승인일
% @param   year,month,day 연,월,일 순으로 입력
\approvaldate{2020}{06}{25}

% @command refereedate 심사위원논문심사일
% @param   year,month,day 연,월,일 순으로 입력
\refereedate{2020}{06}{25}

% @command gradyear 졸업년도
\gradyear{2020}

% 본문 시작
\begin{document}

    % 앞표지, 속표지, 학위논문 제출승인서, 학위논문 심사완료 검인서는
    % 클래스 옵션을 final로 지정해주면 자동으로 생성되며,
    % 반대로 옵션을 draft로 지정해주면 생성되지 않습니다.

    % 영문초록 (abstract)
    \begin{abstract}

    \end{abstract}

    % 목차 (Table of Contents) 생성
    \tableofcontents

    % 표목차 (List of Tables) 생성
%    \listoftables

    % 그림목차 (List of Figures) 생성
    \listoffigures

    % 위의 세 종류의 목차는 한꺼번에 다음 명령으로 생성할 수도 있습니다.
    %\makecontents

%% 이하의 본문은 LaTeX 표준 클래스 report 양식에 준하여 작성하시면 됩니다.
%% 하지만 part는 사용하지 못하도록 제거하였으므로, chapter가 문서 내의
%% 최상위 분류 단위가 됩니다.
%% You cannot use 'part'

1 ABCDEFGHIJKLMNOPQRSTUVWXYZ abcdefghijklmnopqrstuvwxyz \\
2 ABCDEFGHIJKLMNOPQRSTUVWXYZ abcdefghijklmnopqrstuvwxyz \\
3 ABCDEFGHIJKLMNOPQRSTUVWXYZ abcdefghijklmnopqrstuvwxyz \\
4 ABCDEFGHIJKLMNOPQRSTUVWXYZ abcdefghijklmnopqrstuvwxyz \\
5 ABCDEFGHIJKLMNOPQRSTUVWXYZ abcdefghijklmnopqrstuvwxyz \\
6 ABCDEFGHIJKLMNOPQRSTUVWXYZ abcdefghijklmnopqrstuvwxyz \\
7 ABCDEFGHIJKLMNOPQRSTUVWXYZ abcdefghijklmnopqrstuvwxyz \\
8 ABCDEFGHIJKLMNOPQRSTUVWXYZ abcdefghijklmnopqrstuvwxyz \\
9 ABCDEFGHIJKLMNOPQRSTUVWXYZ abcdefghijklmnopqrstuvwxyz \\
000 ABCDEFGHIJKLMNOPQRSTUVWXYZ abcdefghijklmnopqrstuvwxyz \\
1 ABCDEFGHIJKLMNOPQRSTUVWXYZ abcdefghijklmnopqrstuvwxyz \\
2 ABCDEFGHIJKLMNOPQRSTUVWXYZ abcdefghijklmnopqrstuvwxyz \\
3 ABCDEFGHIJKLMNOPQRSTUVWXYZ abcdefghijklmnopqrstuvwxyz \\
4 ABCDEFGHIJKLMNOPQRSTUVWXYZ abcdefghijklmnopqrstuvwxyz \\
5 ABCDEFGHIJKLMNOPQRSTUVWXYZ abcdefghijklmnopqrstuvwxyz \\
6 ABCDEFGHIJKLMNOPQRSTUVWXYZ abcdefghijklmnopqrstuvwxyz \\
7 ABCDEFGHIJKLMNOPQRSTUVWXYZ abcdefghijklmnopqrstuvwxyz \\
8 ABCDEFGHIJKLMNOPQRSTUVWXYZ abcdefghijklmnopqrstuvwxyz \\
9 ABCDEFGHIJKLMNOPQRSTUVWXYZ abcdefghijklmnopqrstuvwxyz \\
000 ABCDEFGHIJKLMNOPQRSTUVWXYZ abcdefghijklmnopqrstuvwxyz \\
1 ABCDEFGHIJKLMNOPQRSTUVWXYZ abcdefghijklmnopqrstuvwxyz \\
2 ABCDEFGHIJKLMNOPQRSTUVWXYZ abcdefghijklmnopqrstuvwxyz \\
3 ABCDEFGHIJKLMNOPQRSTUVWXYZ abcdefghijklmnopqrstuvwxyz \\
4 ABCDEFGHIJKLMNOPQRSTUVWXYZ abcdefghijklmnopqrstuvwxyz \\
5 ABCDEFGHIJKLMNOPQRSTUVWXYZ abcdefghijklmnopqrstuvwxyz \\
6 ABCDEFGHIJKLMNOPQRSTUVWXYZ abcdefghijklmnopqrstuvwxyz \\
7 ABCDEFGHIJKLMNOPQRSTUVWXYZ abcdefghijklmnopqrstuvwxyz \\
8 ABCDEFGHIJKLMNOPQRSTUVWXYZ abcdefghijklmnopqrstuvwxyz \\
9 ABCDEFGHIJKLMNOPQRSTUVWXYZ abcdefghijklmnopqrstuvwxyz \\
000 ABCDEFGHIJKLMNOPQRSTUVWXYZ abcdefghijklmnopqrstuvwxyz \\
1 ABCDEFGHIJKLMNOPQRSTUVWXYZ abcdefghijklmnopqrstuvwxyz \\
2 ABCDEFGHIJKLMNOPQRSTUVWXYZ abcdefghijklmnopqrstuvwxyz \\
3 ABCDEFGHIJKLMNOPQRSTUVWXYZ abcdefghijklmnopqrstuvwxyz \\
4 ABCDEFGHIJKLMNOPQRSTUVWXYZ abcdefghijklmnopqrstuvwxyz \\
5 ABCDEFGHIJKLMNOPQRSTUVWXYZ abcdefghijklmnopqrstuvwxyz \\
6 ABCDEFGHIJKLMNOPQRSTUVWXYZ abcdefghijklmnopqrstuvwxyz \\
7 ABCDEFGHIJKLMNOPQRSTUVWXYZ abcdefghijklmnopqrstuvwxyz \\
8 ABCDEFGHIJKLMNOPQRSTUVWXYZ abcdefghijklmnopqrstuvwxyz \\
9 ABCDEFGHIJKLMNOPQRSTUVWXYZ abcdefghijklmnopqrstuvwxyz \\
000 ABCDEFGHIJKLMNOPQRSTUVWXYZ abcdefghijklmnopqrstuvwxyz \\
Cite example\cite{Moin1998}

%%
%% 한글요약문 시작 (Korean summary)
%%
%% Note. 영문논문일 경우에만 필요하니 한글논문의 경우에는 작성하지 마십시오.
%%
\begin{summarykorean}

\end{summarykorean}

%%
%% 참고문헌 시작
%% Refences
%%

\bibliographystyle{unsrt}
\bibliography{mybib_ryu}

%%
%% 감사의 글 시작
%% Acknowledgement
%%
% @command acknowledgement 감사의글
% @options [default: 클래스 옵션 korean|english ]
% - korean : 한글타이틀 | english : 영문타이틀

\acknowledgement[korean]


    %%때로는 엄하고 때로는 부드러운 모습으로 지도 해주신 한준희 교수님, 감사드립니다.
%%
%% 이력서 시작
%% Curriculum Vitae
%%
% @command curriculumvitae 이력서
% @options [default: 클래스 옵션 korean|english ]
% - korean : 한글이력서 | english : 영문이력서
\curriculumvitae[korean]

    % @environment personaldata 개인정보
    % @command     name         이름
        % input data only you want
    \begin{personaldata}
        \name       {Sae Hee Ryu}
    \end{personaldata}

    % @environment education 학력
    % @options [default: (none)] - 수학기간을 입력
    \begin{education}
%	\item[2013. 9.\ --\ 2020. 8.] Department of Physics, POSTECH (Ph.D.)
%	\item[2009. 3.\ --\ 2013. 8.] Department of Physics, POSTECH (B.S.)
	\item[2013. 9.\ --\ 2020. 8.] Department of Physics, Pohang University of Science and Technology (Ph.D.)
	\item[2009. 3.\ --\ 2013. 8.] Department of Physics, Pohang University of Science and Technology (B.S.)
    \end{education}

    % @environment experience 경력
    % @options [default: (none)] - 해당기간을 입력
   \begin{experience}
	\item[2016. 8.\ --\ 2019. 9.] Technical Research Personnel, Alternative military service
	\item[2014. 3.\ --\ 2015. 9.] Developing endstation for spin-ARPES beamline in PAL beamline 4A2 (Institute for Basic Science)
   \end{experience}

    % @environment activity 학회활동
    % @options [default: (none)] - 활동내용을 입력
%    \begin{affiliation}

%    \end{affiliation}

    \afterpage{\blankpage}  % 마지막장 백색별지

%% 본문 끝
\end{document}
