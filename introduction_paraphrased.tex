Event-time data, ubiquitous across domains like social media \cite{seismic, snapnets}, healthcare \cite{bib:RMTPP, meng2023dynamic}, and stock market trends \cite{Bacry2014MarketIA, hawkesInFinance}, 
consist of sequences of events occurring at specific times and often categorized by type. 
To model such temporal data as a stochastic process, the goal is to predict the time and type of future events based on prior observations, i.e., the history of the sequence. 
For instance, analyzing a person’s purchase history may help forecast when they will make their next purchase, while a brief post from an influential figure on social media could trigger significant movements in stock prices. 
This prediction often translates to estimating the likelihood of an event $e$ occurring at time $t$, represented as a probability density function (pdf) $f(e, t)$. 

Temporal Point Processes (TPP) have traditionally been used for such event predictions, where a realization of the stochastic process is a sequence of ordered time points \cite{lec:tpp}. 
By treating the event times as random variables, TPPs model the occurrence times of events based on past timestamps. 
The Hawkes Process, a well-known TPP model, captures these dynamics by employing a \textit{conditional intensity function} that aggregates excitation effects from past events, a characteristic known as ``self-excitation'' \cite{bib:hawkesOrigin,bib:hawkes}. 
This framework independently models the influence of each event and combines them using a simple summation, making it both interpretable and computationally efficient. 

Although many TPP-based models effectively handle the dynamic prediction of future events, 
they often overlook details such as spatial information, node-specific attributes (particularly in graph-based contexts like social or sensor networks), and additional context such as text, tokens, or images. 
Marked Temporal Point Processes (MTPP) address this limitation by introducing ``marks,'' which represent event types, extending the traditional univariate TPP to account for multivariate event types. 
Recent MTPP approaches frequently combine the intensity functions of Hawkes Processes with neural networks to flexibly model the pdfs of marks \cite{bib:nhp, bib:NJSDE, bib:sahp, bib:STPP, bib:ANHP}. 
Prominent examples include Recurrent Marked Temporal Point Process \cite{bib:RMTPP}, Neural Hawkes Process \cite{bib:nhp}, and Transformer Hawkes Process \cite{bib:THP}, which effectively capture MTPP dynamics. 

While these models excel at capturing complex dependencies, they often sacrifice interpretability of the relationships between events. 
Many existing approaches either directly estimate the joint probability distribution over time and marks, or use a proxy intensity function that does not fully capture the dynamic interaction between different marks and timestamps. 
Additionally, reconstructing the mark pdf or other related functions like the survival function typically involves complex integrations without closed-form solutions, requiring re-computation for each new event during inference. This results in computational inefficiencies, often scaling exponentially with the number of events.

In this context, we introduce a novel framework for MTPP modeling called Dec-ODE, characterized by decoupled hidden state dynamics and driven by latent continuous dynamics via neural ordinary differential equations (ODEs). 

Dec-ODE decouples the hidden state dynamics for each event, allowing the model to independently quantify each event’s contribution to the overall stochastic process with high fidelity. 
Moreover, this decoupling leverages ODEs to enable parallel training of continuous hidden state dynamics, significantly reducing computational overhead compared to sequential approaches that rely on step-by-step event updates.

Our Dec-ODE formulation offers the following contributions:
\begin{itemize}
    \item Dec-ODE decouples individual events within a stochastic process for efficient learning and explainability of each mark, 
    \item Dec-ODE models the continuous dynamics of influences from individually decoupled events by leveraging Neural ODEs,
    \item Dec-ODE demonstrates state-of-the-art performance on various benchmarks for MTPP validation while effectively capturing inter-time dynamics with interpretability. 
\end{itemize} 