사건의 종류와 발생 시간의 쌍으로 이루어진 데이터는 의료, 산업, 금융 등 다양한 분야에서 수집되고 사용되고있다.
이러한 시계열 데이터를 확률론적 과정으로 모델링을 한다면 정보를 분석하거나 미래에 발생할 정보들을 예측하는 것에 효과적으로 사용될 수 있다.
특히 사건의 발생 시간에 대한 예측을 위해 자주 사용되는 방법론으로는 Marked Temporal Point Process (MTPP) 가 있다.

이러한 MTPP를 인공지능을 활용하여 모델링하는 것에는 다양한 접근 방법이 제안되었다.
이중에서 대표적인 접근 방법들로는 RNN과 Transformer기반의 모델을 사용하는 방법들이 성능적인 측면에서 효과를 입증하였다.
특히 이러한 방법론들은 사건들 사이의 관계성을 학습하여 효과적인 예측을 하는 것을 장점으로하지만, 
모델이 어떠한 이유로 이러한 결과를 출력하였는지에 대한 설명성을 잃는 경우가 많다.
하지만 실제 생활에서는 다음 사건의 예측 뿐 아니라, 영향력에 대한 분석, 생존분석, 그리고 사건의 확률 분포에 대한 이해 등 다양한 활용이 요구된다.
다양한 분야에 활용되기 사건의 개별적인 영향력을 파악하는 것은 큰 중요성을 갖는다.

이러한 문제점을 해결하고자 본 논문은 이벤트들의 개별적인 영향력을 표현하는 프레임워크를 제안하였다.
각 영향력은 상미분방정식을 통해 모델링되고, MTPP에 대한 특성을 미분방정식의 특성을 활용하여 효율적으로 계산을 하는 방법론 또한 함께 제안하였다.
마지막으로 제안된 프레임워크의 효과성을 입증하기 위하여 단순한 구조를 갖는 모델과, 해당 모델을 활용한 더 효율적인 학습 방법을 제안하였다.

이벤트의 영향력을 설명가능하게 표현하기 위해 각각의 이벤트는 독립적인 시간에 따라 변화하는 프로세스로 표현된다.
이때 보다 효율적인 계산을 위해 다차원 프로세스를 상미분방적식으로 표현한다.
이 프로세스는 이벤트의 발생 시간에 대한 확률 분포를 계산하기 위한 $\mu$와 이벤트의 종류의 분포를 계산하기 위한 $\hat{f}$로 변환되게 된다.
시간에 대한 효율적인 예측을 위해 확률분포, 생존률 등의 예측값들의 시간에 따른 적분 값들은 미분방정식을 활용하여 프로세스와 함께 계산한다.
이렇게 독립적으로 모델링을 하는 것은 예측을 하는 것에 원하는 이벤트의 집합을 고려할 수 있는 장점을 제공한다.

이러한 방법론의 효과성을 보이기 위해 과거의 영향을 선형 합산 등 단순한 함수를 통해 관심 시점의 값을 계산하는 구현을 제안하였다.
이렇게 단순한 구현하는 것의 장점으로는 학습시 적분을 위해 모든 시작 시점부터 순차적으로 계산을 하는 것이 아니라, 
각각의 이벤트를 자신의 시점에 맞추어 동시에 계산 후 추후 손실함수 계산을 위해 사용 될 수 있다.

해당 프레임워크는 설명가능하고 연속적인 MTPP의 모델링을 하는 것 뿐 아니라, 
최고 성능을 내는 기존 방법론보다 좋거나 유사한 성능을 보이는 것을 보였다.
또한 제안된 방법론의 효율성 및, 프레임워크의 확장성이 실험을 통해 확인 되었다.

\